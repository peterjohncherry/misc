\documentclass[12pt]{article}
\usepackage[utf8x]{inputenc}
\usepackage[english]{babel}
\usepackage[T1]{fontenc}
\usepackage{wrapfig}
\usepackage{amsmath}
\usepackage{amssymb}
\usepackage[font=small,format=plain,labelfont=bf,up,textfont=it,up]{caption}
\usepackage{calc}
\usepackage{ifthen}
\usepackage{relsize}
\usepackage{bbold}
\usepackage{mathtext}
\usepackage{pdfpages}
\usepackage{geometry}
 \geometry{
 a4paper,
 total={170mm,257mm},
 left=35mm,
 top=25mm,
 bottom=25mm,
 right=20mm,
 }

\begin{document}
\noindent \emph{Note : Adapted from B. Martin, J. Autschbach, J. Chem. Phys., 142(5), 054108, 2015 } \\

\noindent The expression for the total shielding for a triplet is
\begin{equation}
\sigma = \sigma^{orb} - \frac{\beta_{e}}{g_{N}\beta_{N}}\frac{1}{k_{B}T} \mathbf{g}\mathbf{Z}\mathbf{A}
\label{eqn:totalshieldingtriplet}
\end{equation}
where $\sigma^{orb}$ is the orbital shielding (defined as the only non-zero contribution to the shielding for a diagmagnetic molecule), $\beta_{e}$ the Bohr magneton, $\beta_{N}$ the nuclear magneton,
$k_{B}$ the Boltzman constant, $T$ the temperature, $\mathbf{g}$ the g-matrix, $\mathbf{A}$ the hyperfine matrix, and $\mathbf{Z}$ the Z-matrix, the
definition of which shall be discussed below.\\

\noindent It is assumed that $\mathbf{g}$ and $\mathbf{Z}$ have the same principle axes, however, the hyperfine tensor $\mathbf{A}$
may have different principal axes, i.e.,
\begin{equation}
\mathbf{g}\mathbf{Z}\mathbf{A} = 
\begin{bmatrix}
g_{\perp} & 0          & 0 \\
0         & g_{\perp}  & 0 \\
0         & 0          & g_{\parallel} 
\end{bmatrix}
\begin{bmatrix}
Z_{\perp} & 0          & 0 \\
0         & Z_{\perp}  & 0 \\
0         & 0          & Z_{\parallel} 
\end{bmatrix}
\begin{bmatrix}
A_{11} & A_{12}  & A_{13} \\
A_{21} & A_{22}  & A_{23} \\
A_{31} & A_{32}  & A_{33} 
\end{bmatrix}
\end{equation}
The expression for the isotropic contributions to the paramagnetic shielding is split up into a contact and pseudocontact part; 
$\sigma_{iso}^{P} = \sigma_{iso}^{c}+\sigma_{iso}^{pc}$, which may be written as 
\begin{equation}
\sigma_{iso}^{c} = \frac{\beta_{e}A_{iso}}{3 T g_{n} \beta_{N} }[g_{iso}(Z_{\parallel}+2Z_{\perp})+2\Delta g (Z_{\parallel}-Z_{\perp})]
\end{equation}
\noindent In an axial system the energies can be shifted so that when represented in the principal axes the zero-field splitting matrix has the form
\begin{equation}
\begin{bmatrix}
0 & 0  & 0 \\
0         & 0  & 0 \\
0         & 0  & D
\label{eqn:axiald}
\end{bmatrix}
\end{equation}
\noindent The energies of the three states in the triplet are then $\{D, 0, D\}$. The components of $\mathbf{Z}$ are then given by
\begin{equation}
Z_{\parallel} = \frac{2 e^{-\beta D}}{1+2e^{-\beta D}}
\label{eqn:zparalleltriplet}
\end{equation}
\begin{equation}
Z_{\perp} = \frac{\frac{2}{\beta D} (1-e^{-\beta D})}{1+2e^{-\beta D}}
\label{eqn:zperptriplet}
\end{equation}
Here, $D$ is the energy gap between the singlet and doublet single states which comprise the triplet.
The denominator is the partition function. The sign of $D$ can in principal be positive or negative. However, in such cases it is perhaps
simpler to chose an energy shift such that the energy of the states is $\{0, D, 0\}$, leading to the following expressions for components of $\mathbf{Z}$
\begin{equation}
Z_{\parallel} = \frac{2}{2+e^{-\beta D}}
\label{eqn:zparalleltripletnegative}
\end{equation}
\begin{equation}
Z_{\perp} = \frac{\frac{2}{\beta D} (e^{-\beta D} -1 )}{2+e^{-\beta D}}
\label{eqn:zperptripletnegative}
\end{equation}
With these definitions 
\noindent Using these expressions, expanding the exponential in powers of T, and only taking terms of $\mathcal{O}(T^{3})$ and lower
yields the following expressions for the contact and pseudo-contact contributions to the paramagnetic shielding:
\begin{equation}
\sigma^{c}_{iso} = C_{iso}A_{iso}  \left[ 2g_{iso}\beta - \frac{2}{3}D\Delta g\beta^{2} - \frac{1}{9}D^{2}g_{iso}\beta^{3} \right]
\label{eqn:contacttriplet}
\end{equation}
\begin{equation}
\sigma^{pc}_{iso} = C_{iso}\Delta A_{iso} \left{[} 4 \Delta g\beta - \frac{2}{3}D(g_{iso}+\Delta g) \beta^{2} - \frac{2}{9}D^{2}\Delta g_{iso}\beta^{3}
\label{eqn:pseudocontacttriplet}
\end{equation}



\noindent For cases where the rhombic parameter, $E$, is non-zero, the orientation and energy shift may be chosen such that the zfs matrix 
has the form
\begin{equation}
\begin{bmatrix}
\frac{-D}{3}+E & 0               & 0 \\
0              & \frac{-D}{3}-E  & 0 \\
0              & 0               & \frac{2D}{3}
\end{bmatrix}
\label{eqn:rhombicd}
\end{equation}
Using a similar logic as to derive the expressions in the axial case, the components of $\mathbf{Z}$ for a rhombic 
system are given by
\begin{equation}
Z_{11} = \frac{\frac{2}{\beta (D+E)}(1-e^{\beta (D+E)})}
{1+e^{-\beta (D-E)}+ e^{-\beta (D+E)} }
\label{eqn:z11rhombic}
\end{equation}
\begin{equation}
Z_{22} = \frac{\frac{2}{\beta (D-E)}(1-e^{\beta (D-E)})}
{1+e^{-\beta (D-E)}+ e^{-\beta (D+E)} }
\label{eqn:z22rhombic}
\end{equation}
\begin{equation}
Z_{33} = \frac{\frac{1}{\beta E}(e^{\beta (D-E)} - e^{\beta (D+E)})}
{1+e^{-\beta (D-E)}+ e^{-\beta (D+E)} }
\label{eqn:z33rhombic}
\end{equation}


\end{document}
