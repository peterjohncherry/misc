\documentclass[12pt]{article}
\usepackage[utf8x]{inputenc}
\usepackage[english]{babel}
\usepackage[T1]{fontenc}
\usepackage{wrapfig}
\usepackage{amsmath}
\usepackage{amssymb}
\usepackage[font=small,format=plain,labelfont=bf,up,textfont=it,up]{caption}
\usepackage{calc}
\usepackage{ifthen}
\usepackage{relsize}
\usepackage{bbold}
\usepackage{mathtext}
\usepackage{pdfpages}
\usepackage{geometry}
 \geometry{
 a4paper,
 total={170mm,257mm},
 left=35mm,
 top=25mm,
 bottom=25mm,
 right=20mm,
 }

\begin{document}
\section{Appendix}
First, the spin operators represented in the spin product basis (these are the
matrices used by EasySpin).
\begin{equation}
S(x,1) = 
\begin{bmatrix}
0                  & \frac{1}{\sqrt{2}} & 0\\
\frac{1}{\sqrt{2}} & 0                 & \frac{1}{\sqrt{2}} \\ 
0                  & \frac{1}{\sqrt{2}} & 0
\end{bmatrix}
\end{equation}
\begin{equation}
S(y,1) = 
\begin{bmatrix}
0                  & \frac{i}{\sqrt{2}} & 0\\
\frac{-i}{\sqrt{2}} & 0                 & \frac{-i}{\sqrt{2}} \\ 
0                  & \frac{i}{\sqrt{2}} & 0
\end{bmatrix}
\end{equation}
\begin{equation}
S(z,1) = 
\begin{bmatrix}
-1                  & 0 & 0\\
0 & 0               & 0    \\ 
0                   &   & 1
\end{bmatrix}
\end{equation}
\noindent Now for the product operators, $\hat{s}_{i}\otimes\hat{s}_{j}$,
represented in the spin product basis.
\begin{equation}
S_{xx} = 
\begin{bmatrix}
\frac{1}{2} & 0  & \frac{1}{2} \\ 
0           & 1 & 0\\
\frac{1}{2} & 0  & \frac{1}{2} 
\end{bmatrix}
\end{equation}

\begin{equation}
s_{xy} = 
\begin{bmatrix}
\frac{-i}{2} & 0  & \frac{i}{2} \\ 
0           & 1 & 0\\
\frac{-i}{2} & 0  & \frac{i}{2} 
\end{bmatrix}
\end{equation}

\begin{equation}
S_{xz} = 
\begin{bmatrix}
0           & 0 & 0\\
\frac{1}{2} & 0  & \frac{1}{2} \\ 
0           & 0 & 0\\
\end{bmatrix}
\end{equation}


\begin{equation}
s_{yx}
\begin{bmatrix}
\frac{i}{2} & 0  & \frac{i}{2} \\ 
0           & 1 & 0\\
\frac{-i}{2} & 0  & \frac{-i}{2} 
\end{bmatrix}
\end{equation}

\begin{equation}
s_{yy} = 
\begin{bmatrix}
0           & 0 & 0\\
\frac{1}{2} & 0  & \frac{1}{2} \\ 
0           & 0 & 0\\
\end{bmatrix}
\end{equation}
\begin{equation}
s_{yz} = 
\begin{bmatrix}
0           & 0 & 0\\
\frac{1}{2} & 0  & \frac{1}{2} \\ 
0           & 0 & 0\\


\frac{1}{2}  & 0  & \frac{-1}{2} \\ 
0            & 1  & 0\\
\frac{-1}{2} & 0  & \frac{1}{2} 
\end{bmatrix}
\end{equation}

\begin{equation}
s_{zx}
\begin{bmatrix}
\frac{i}{2} & 0  & \frac{i}{2} \\ 
0           & 1 & 0\\
\frac{-i}{2} & 0  & \frac{-i}{2} 
\end{bmatrix}
\end{equation}

\begin{equation}
s_{zy} = 
\begin{bmatrix}
0     &  \frac{-i}{\sqrt{2}} & 0 \\ 
0     & 0                    & 0 \\
0     &  \frac{-i}{\sqrt{2}} & 0 \\
\end{bmatrix}
\end{equation}
\begin{equation}
s_{zz} = 
\begin{bmatrix}
1     & 0 & 0\\
0     & 0 & 0 \\ 
0     & 0 & 1\\
\end{bmatrix}
\end{equation}



\end{document}
