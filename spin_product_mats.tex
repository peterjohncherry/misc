\documentclass[12pt]{article}
\usepackage[utf8x]{inputenc}
\usepackage[english]{babel}
\usepackage[T1]{fontenc}
\usepackage{wrapfig}
\usepackage{amsmath}
\usepackage{amssymb}
\usepackage[font=small,format=plain,labelfont=bf,up,textfont=it,up]{caption}
\usepackage{calc}
\usepackage{ifthen}
\usepackage{relsize}
\usepackage{bbold}
\usepackage{mathtext}
\usepackage{pdfpages}
\usepackage{geometry}
 \geometry{
 a4paper,
 total={170mm,257mm},
 left=35mm,
 top=25mm,
 bottom=25mm,
 right=20mm,
 }

\begin{document}
\noindent Suppose we have two unpaired electrons. The two-particle Hilbert
space spanned by

\begin{equation} |s_{1}m_{1}\rangle \otimes | s_{2}m_{2} \rangle = |
s_{1}m_{1}, s_{2}m_{2} \rangle,
\end{equation}

\noindent the RHS is just a short hand of writing one of the vectors formed
from the product $\otimes$. From now on this will be referred to as the
\emph{product basis} . \\

\noindent The particles have spin operators denoted $\hat{S}_{1}$ and
$\hat{S}_{2}$. The product states satisfy

\noindent As $m_{1}  \text{and} m_{2} = \pm \frac{1}{2} $, we shall use the
further shorthand, e.g., the product state for $s_{1,2} = \frac{1}{2}$, $m_{1}
= +\frac{1}{2}$ and $m_{2} = -\frac{1}{2} $ will be written
\begin{equation}
|+-\rangle = | \frac{1}{2}\frac{1}{2},\frac{1}{2}\frac{-1}{2} \rangle
\end{equation}
\begin{equation}
S^{2}_{i}|s_{1}m_{1}, s_{2}m_{2} \rangle = \hbar^{2}s_{i}(s+1)|s_{1}m_{1}, s_{2}m_{2} \rangle
\end{equation}
\noindent The total angular momentum operator, $\mathbf{S}$, is defined 
\begin{equation}
\mathbf{S} = \mathbf{S}_{1}+\mathbf{S}_{2}
\end{equation}
We want to find the eigenvalues of $S_{z}$ and $S^{2}$ in this basis. First, look at
$S_{z}$.
\begin{equation}
S_{z}|++\rangle =
(S_{1z}+S_{2z})|++ = (\frac{\hbar}{2} + \frac{\hbar}{2}) | ++ rangle = |++\rangle
\end{equation}
\begin{equation}
S_{z}|++\rangle =
(S_{1z}+S_{2z})|++ = (\frac{-\hbar}{2} + \frac{\hbar}{2}) | -+ rangle = 0
\end{equation}
\begin{equation}
S_{z}|++\rangle =
(S_{1z}+S_{2z})|++ = (\frac{\hbar}{2} + \frac{-\hbar}{2}) | +- rangle = 0
\end{equation}
\begin{equation}
S_{z}|++\rangle = 
(S_{1z}+S_{2z})|++ = (\frac{-\hbar}{2} + \frac{-\hbar}{2}) | -- \rangle = | -- \rangle
\end{equation}
So the representation of $\mathbf{S}_{z}$ in the product basis is
\begin{equation}
[S_{z}]_{\{|1,2\rangle\}} = 
\begin{bmatrix}
1                  & 0 & 0 & 0 \\
0                  & 0 & 0 & 0 \\
0                  & 0 & 0 & 0 \\
0                  & 0 & 0 & -1 
\end{bmatrix}
\end{equation}
where the $\hbar$ has been omitted for convenience.\\

\noindent Now consider the $S^{2}$ operator. 
\begin{equation}
S^{2} = (\mathbf{S}_{1}+\mathbf{S}_{2})\cdot (\mathbf{S}_{1}+\mathbf{S}_{2})= 
S^{2}_{1}+S^{2}_{2} + 2\mathbf{S}_{1}\cdot \mathbf{S}_{2}
\end{equation}
\noindent Unfortunately, the product states are not eigenfunctions of $S^{2}$.
This leads us to construct the following basis:

\begin{equation*}
\frac{|+-\rangle - | -+ \rangle}{\sqrt{2}}
\end{equation*}
\\
\begin{equation*}
|++\rangle
\end{equation*}
\begin{equation*}
\frac{|+-\rangle - | -+ \rangle}{\sqrt{2}}
\end{equation*}
\begin{equation*}
|--\rangle
\end{equation*}
\noindent I'll refer to this as the "ST basis". The question is now which of
these two bases should we use. If we are interested in and operator,
$\hat{V}^{(+)}$ of the form,
\begin{equation}
\hat{V}^{(+)} = -(v^{(+)}_{1}\mathbf{S}_{1} + v^{(+)}_{2}\mathbf{S}_{2}) \cdot \mathbf{B}
\end{equation}
this indicates non-interacting spins, hence it makes sense to use the product
basis.  However, if we have and operator $\hat{V}^{\times}$, corresponding to 
interacting spins
\begin{equation}
\hat{V}^{\times} = v^{\times}\mathbf{S}_{1}\cdot\mathbf{S}_{2}
\end{equation}
then it makes send to use the ST basis.\\

\noindent Soncini provides a basis of matrices which may be used to represent
the representation of an operator in the above bases. To understand how this
basis is constructed, it is useful to recap the addition of angular momentum,
and how bases such as those above can be transformed into one another. 

\section{Addition of classical angular momentum}
Consider adding two angular momenta, and the associated product states. As
with spin we have
\begin{equation}
J_{z}|j_{1}m_{1} \rangle = (m_{1}+m_{2})|j_{1}m_{1}, j_{2}m_{2} \rangle
\end{equation}
Note that his is going to be degenrate, as multiple combinations of $m_{1}$ and
$m_{2}$ can result in the same eigenvalue. Within each of these degenerate
eigenspaces (which can be thought of corresponding to spin multiplets) we want
to chose a basis which consists solely of eigenvectors of $J^{2}$.\\

\noindent If $j_{1}\geq j_{2}$, then the value of the total angular momentum $j$ can
take on values 
\begin{equation}
(j_{1}+j_{2}-1), (j_{1}+j_{2}-2), ... , (j_{1}-j_{2})
\end{equation}
\noindent So the total number of states in this basis is $(2j_{1}+1)(2j_{2}+1)$.
In fact we can state that the tensor product space $j_{1}\otimes j_{2}$ can
be represented as a sum of spaces, each of which is associated with one the 
eigenvalues stated above, i.e.,
\begin{equation}
j_{1}\otimes j_{2} =
(j_{1}+j_{2})\oplus
(j_{1}+j_{2}-1)\oplus
...
(j_{1}-j_{2})
\end{equation}
The states within each one of these degenerate subspaces are
\begin{equation}
|JM, j_{1}j_{2} \rangle 
\text{ \ \ where \ \ } (j_{1}+j_{2}) \geq j \geq (j_{1}-j_{2})
\text{ \ \ and \ \  } j\geq m \geq -j
\end{equation}
Now we can consider which states exist in each one of these degenerate subspaces
\begin{table}[]
\begin{tabular}{@{}llllll@{}}
\hline
\multicolumn{1}{l||}{}     & \multicolumn{1}{l|}{$J=j_{1}+j_{2}$} & \multicolumn{1}{l|}{$J=j_{1}+j_{2}-1$}      & \multicolumn{1}{l|}{$J=j_{1}+j_{2}-2$} & \multicolumn{1}{l}{$J=j_{1}+j_{2}-3$} \\
\hline\hline
\multicolumn{1}{l||}{m=3}  & \multicolumn{1}{l|}{$|3,3\rangle$}  &  \multicolumn{1}{l|}{} & \multicolumn{1}{l|}{} & \multicolumn{1}{l}{}               & \multicolumn{1}{l}{}              \\
\multicolumn{1}{l||}{m=2}  & \multicolumn{1}{l|}{$|3,2\rangle $} &  \multicolumn{1}{l|}{$|2,2\rangle$ }           & \multicolumn{1}{l|}{}               & \multicolumn{1}{l}{}              \\
\multicolumn{1}{l||}{m=1}  & \multicolumn{1}{l|}{$|3,1\rangle $} &  \multicolumn{1}{l|}{$|2,1\rangle$ }           & \multicolumn{1}{l|}{$|1,1\rangle $} & \multicolumn{1}{l}{}              \\
\multicolumn{1}{l||}{m=0}  & \multicolumn{1}{l|}{$|3,0\rangle $} &  \multicolumn{1}{l|}{$|2,0\rangle$ }           & \multicolumn{1}{l|}{$|1,0\rangle $} & \multicolumn{1}{l}{$|0,0\rangle $}  \\
\multicolumn{1}{l||}{M=-1} & \multicolumn{1}{l|}{$|3,-1\rangle$} & \multicolumn{1}{l|} {$|2,-1\rangle$ }          & \multicolumn{1}{l|}{$|1,-1\rangle$} & \multicolumn{1}{l}{}              \\
\multicolumn{1}{l||}{M=-2} & \multicolumn{1}{l|}{$|3,-2\rangle$} & \multicolumn{1}{l|} {$|2,-2\rangle$ }          & \multicolumn{1}{l|}{}               & \multicolumn{1}{l}{}              \\
\multicolumn{1}{l||}{M=-3} & \multicolumn{1}{l|}{$|3,-3\rangle$} & \multicolumn{1}{l|}{} & \multicolumn{1}{l|}{}  & \multicolumn{1}{l}{}               & \multicolumn{1}{l}{} 
\end{tabular}
\end{table}
Or, when applied to the $j_{1} = j_{2} = \frac{1}{2}$ case we have
\begin{table}[]
\begin{tabular}{@{}llllll@{}}
\hline
\multicolumn{1}{l||}{}     &   \multicolumn{1}{l|}{$J= 1 $} & \multicolumn{1}{l}{$J=0$} \\
\hline\hline
\multicolumn{1}{l||}{m=1}  &    \multicolumn{1}{l|}{$|1,1\rangle $} & \multicolumn{1}{l}{}              \\
\multicolumn{1}{l||}{m=0}  &    \multicolumn{1}{l|}{$|1,0\rangle $} & \multicolumn{1}{l}{$|0,0\rangle $}  \\
\multicolumn{1}{l||}{M=-1} &     \multicolumn{1}{l|}{$|1,-1\rangle$} & \multicolumn{1}{l}{}              \\
\end{tabular}
\end{table}
It must be possible to obtain these states (which have a good value of $J^{2}$)
via some transformation of the states in the product basis. We can see that for
the $|S=1,M=1\rangle$ state in the product basis is the same as the state in
the ST basis. We can now generate the other states via use of the ladder
operators on this high spin state, i.e.,

\begin{equation*}
\hat{S}_{-} | 1,1\rangle = \frac{1}{\sqrt{2}}|1,0\rangle 
\end{equation*}
\begin{equation*}
= (\hat{s}_{1-}+\hat{s}_{2-}) | 1,1\rangle = \frac{1}{\sqrt{2}}(|+-\rangle + |-+\rangle)
\end{equation*}
acting again with the ladder operator yields
\begin{equation}
S_{-1}|S=1,M=0\rangle  = |--\rangle
\end{equation}
\noindent Whilst this proeedure is done for the simple case of two similar
spins of $\frac{1}{2}$, is it can be applied generally to any angualr momenta,
and hence may be applied recursively to build up systems of more than two
couple momenta. For example, the appropriate basis for three coupled spins can
be obtained by first constructing the ST basis for two coupled spins, and then
coupling a further third spin to the states in this ST basis.\\

\noindent A side note is that the choice of phase for these spins can alter the
sign resulting from application of the ladder operators. It is important to
bear this in mind, but generally, the phase is chose so that the signs yielded
are those shown above.\\

\noindent We now need to consider $|S=0, M=0\rangle$. This can be obtained by
noting it must be orthogonal to the above defined states. By convention, it is
also chosen such that it has real coefficients (I don't know if this convention
has any significant ramifications, or requires justificaiton).  We also require
it to be normalized to 1. Orthogonality to $|++\rangle$ and $|--\rangle$ means
the state must be written as some combination

\begin{equation}
|0,0\rangle = \alpha | +-\rangle +\beta| -+\rangle
 \text{\ \ \ where \ \ } \alpha+\beta =0
 \text{ \ \ \ and \ \ \ } \alpha^{2}+\beta^{2} = 1
\end{equation}
Solving for $\alpha$ and $\beta$ yields
\begin{equation}
|0,0\rangle = \frac{1}{\sqrt{2}} (|+-\rangle - | -+\rangle )
\end{equation}
the standard covention is that the coefficient corresponding to the state with
$m_{1} = j_{1}$ positive for the $|J=0, M=0\rangle$ state. Again this doesn't
truly matter, but is important if we want our coefficients to match up with
everyone elses.\\

\noindent We can now think about extending this to the general problem, i.e.,
the case where $j_{1}$ and $j_{2}$ take any value (where we chose the indexes
so $j_{1} \geq j_{2}$).  As before, the state with the highest angular momenta
($J=j_{1}+j_{2}$) is the same as can be built out of a single product ket 
\begin{equation}
|J= (j_{1}+j_{2}), M=(j_{1}+j_{2})\rangle = | j_{1} =j_{1}, m_{1}=j_{1}; j_{2}=j_{2},m_{2}=j_{2} \rangle 
\end{equation}
\noindent  As before we can use the ladder operators to decrease the value of $M$ one step
\begin{equation}
|J= (j_{1}+j_{2}), M=(j_{1}+j_{2})\rangle =
\frac{1}{\hbar}J_{-}|J= (j_{1}+j_{2}), M=(j_{1}+j_{2})\rangle 
\end{equation}
\noindent hence
\begin{equation*}
| j_{1}+j_{2}, j_{1}+j_{2}-1\rangle
=\sqrt{\frac{1}{2(j_{1}+j_{2})}}(J_{1-}+J_{2-})|J,M\rangle
\end{equation*}
%\begin{equation*}
%=\sqrt{\frac{1}{2(j_{1}+j_{2})}}(J_{1-}+J_{2-}) (|j_{1}= j_{1},j_{2}=(j_{2}-1)\rangle + |j_{1}=(j_{1}-1),j_{2}=j_{2}\rangle)
%\end{equation*}
\begin{equation*}
J=| j_{1}+j_{2},M= j_{1}+j_{2}-1\rangle
=\sqrt{\frac{j_{1}}{j_{1}+j_{2}}}  \Bigg|j_{1}= (j_{1}-1) ,j_{2}=j_{2}\Bigg\rangle+
 \sqrt{\frac{j_{2}}{j_{1}+j_{2}}}  \Bigg|j_{1}= j_{1}   ,j_{2}=(j_{2}-1)\Bigg\rangle
\end{equation*}
So we can obtain all the states with $J=j_{1}+j_{2}$, $J \geq  m_{1} \geq -J $
by successively applying the ladder operator in this manner.\\


\noindent Now we must move onto states where $J= j_{1}+j_{2}-1$. There are only
two product kets from which we may construct the coupled basis; those with $|
j_{1}=j_{1}, m_{1}=j_{1}; j_{2}=j_{2}, m_{2} = j_{2}-1\rangle $and  
$| j_{1}=j_{1}, m_{1}=j_{1}-1; j_{2}=j_{2}, m_{2} = j_{2}\rangle. $. By
inspection the high $M$ state for $J=j_{1}+j_{2}-1$ is
\begin{equation}
|J= j_{1}+j_{2}-1, M=j_{1}+j_{2}-1\rangle =
 \sqrt{\frac{j_{1}}{j_{1}+j_{2}}} \Bigg|j_{1}, j_{1}; j_{2}, (j_{2}-1) \Bigg\rangle
-\sqrt{\frac{j_{2}}{j_{1}+j_{2}}} \Bigg|j_{1}, (j_{1}-1); j_{2}, j_{2} \Bigg\rangle
\end{equation}
\noindent To reiterate, the sign is always chosen to be positive for the state
with $m_{1}=j_{1}$.\\

\noindent We continue to proceed in this manner; apply various constraints to
obtain the maximum possible value of $M$ for each set of states with common
$J$, and obtain all other states in that multiplet through application of the
lowering operator.\\

\noindent The above leads to statements such as 
\begin{equation}
\frac{1}{2}\otimes\frac{1}{2}=1\oplus 0,
\end{equation}
i.e., the space necessary to describe two interacting spins with angular
momenta $1/2$ is equivalent to the space formed from the states of a spin with
angular momentum $1$, and a spin with angular momentum $0$. More generally we
can write

\begin{equation}
J_{1}\otimes J_{2} = \oplus_{i} j'_{i}
\end{equation}
seperating out the space into a sum of a number of other spaces can greatly
simplify some of the maths, and allow for the derivation of various selection
rules.

\section{Clebsch Gordon Coefficients}
The product states form a complete basis for representation of the coupled
states, hence we can write

\begin{equation}
|J, M \rangle=  \sum_{m_{1}}\sum_{m_{2}}
|j_{1}m_{1};j_{2},m_{2}\rangle\langle j_{1}m_{1};j_{2},m_{2}|
j_{1}m_{1};j_{2},m_{2}\rangle 
\end{equation}
\noindent where $\rangle\langle
j_{1}m_{1};j_{2},m_{2}|j_{1}m_{1};j_{2},m_{2}\rangle $ are the referred to as
the Clebsch-Gordon coefficients, i.e., the coefficients which take us from the
product basis to the coupled basis.\\
 
\noindent These coefficients have a number of important properties\\
\begin{equation}
\langle j_{1}, m_{1}; j_{2},m_{2} | JM\rangle \neq 0 \text{ \ \ \ \ iff \ \ \ } j_{1}-j_{2} \leq J \leq j_{1}+j_{2}
\end{equation}
This is referred to as the triangle inequality, as it requires that we must be
able to form a triangle with sides $j_{1}$, $j_{2}$  and $J$. 
\begin{equation}
\langle j_{1}, m_{1}; j_{2},m_{2} | JM\rangle \neq 0 \text{ \ \ \ \ iff \ \ \ } m_{1}+m_{2} = M
\end{equation}
There also, by convention, always real, and as stated previously
\begin{equation}
\langle j_{1}, m_{1}; j_{2},J-j_{1} | JM\rangle \geq 0
\end{equation}
We can also note that the Clebsch-Gordon coefficients must obey the relation
\begin{equation}
\langle j_{1}, m_{1}; j_{2},m_{2} | JM\rangle = 
(-1)^{j_{1}+j_{2}-J}
\langle j_{1}, -m_{1}; j_{2}, -m_{2} | JM\rangle
\end{equation}
indicating the matrix whose elements are the Clebsch-Gordon (CG)  coefficients
is antisymmetry, as we would expect given that it corresponds to a real,
unitary transformation.\\

\noindent An illustrative example can be provided by using the CG coefficients
determined in the previous section to construct a transformation matrix:
\begin{equation}
\begin{bmatrix}
|1,1\rangle\\
|1,0\rangle\\
|0,0\rangle \\
|1,-1\rangle
\end{bmatrix}
=
\begin{bmatrix}
1 & 0                   & 0                  & 0 \\
0 & \sqrt{\frac{1}{2}}  &-\sqrt{\frac{1}{2}} & 0 \\
0 & \sqrt{\frac{1}{2}}  &\sqrt{\frac{1}{2}}  & 0 \\
0 & 0                   & 0                  & 1 
\end{bmatrix}
\begin{bmatrix}
|++\rangle \\
|+-\rangle\\
|-+\rangle\\
|--\rangle
\end{bmatrix}
\end{equation}
\noindent With these tools any space constructed from a product of
angular momenta, $j_{1}\otimes\j_{2}$ is reducible to to a linear combination of
other angular momenta spaces, e.g.,
\begin{equation*}
\frac{1}{2}\otimes\frac{1}{2} = 1 \oplus 0
\end{equation*}
\begin{equation}
\frac{1}{2}\otimes\frac{1}{2} \otimes \frac{1}{2}= \frac{1}{2}\otimes (1 \oplus 0)= \frac{3}{2} \oplus \frac{1}{2}\oplus\frac{1}{2}
\end{equation}
\noindent  Hence any operator which is defined on a space formed from the
product of angular momenta can be similarly decomposed. In the context of
magnetic resonance theory this can be accomplished through use of irreducible
tensor operators (ITOs)/spherical tensor operators(STOs)/irreducible spherical
tensor operators(ISTOs) (sadly, all these various names all seem to be in use,
despite the fact that not all irreducible tensor operators are spherical).\\

\section{Cartesian, Irreducible and Spherical Tensor Operators} The most
commonly encountered tensor are Cartesian tensors (CTs), ones whose elements
transform under rotations in the same way we as a vector. That is, if a vector,
$\mathbf{V}$, with elements $V_{i}$ transforms under a rotation $\mathbf{R}$
between two co-ordinate systems labelled with indexes $i$  and $i'$ as
\begin{equation}
V_{i'} = \sum_{i}R_{i'i}V
\end{equation}
then a tensor $\mathbf{T}$ which has components $T_{i}$ when represented on a
space formed from a product of such vectors will transform as
\begin{equation}
T_{i'j'k'....}R_{i'i}R_{j'j}R_{k'k}...T_{ijk...}
\end{equation}
not all tensors need to transform in this manner. The magnetic resonance
(MR)\footnote{shorthand for EPR/NMR/pNMR} tensors (as they are usually defined)
are generally of this type. However, Soncini's theory takes a different
approach, and instead defines the MR tensors as being irreducible spherical
tensor operators (ISTO).  An irreducible spherical tensor (IST) being one whose
elements transform in the same way as spherical harmonics,
$\Upsilon_{l,m}(\theta,\phi)$.\\

\noindent Whilst in principle Soncini's definition is just a matter of
formalism, it has a number of important differences to previous definitions.
The first thing is to clarify the above definition of an IST: An IST of rank
$l$ has a $2l+1$ elements, whose components transform as spherical harmonics
$2l+1$. Hence, only one set of spherical harmonics, those corresponding to
angular momentum $l$, are involved in the definition of a tensor of a given
rank. A consequence of this is that isomorphism between ISTs and CTs only
exists for tensors of ranks 0 and 1. Hence a Catesian tensor of rank 2
\emph{does not} necessarily correspond to a IST of rank-2, or any linear
combination of such ISTs.\\

\noindent To illustrate this consider a real CT of rank 2 with $9$ independent
elements. Then note that an IST of rank 2 has only $2l+1=5$ elements. However,
we can get round this by chosing an appropriate decomposition of the CT tensor,
e.g., \begin{equation} \mathbf{T} = \mathbf{A}+\mathbf{B}+\mathbf{C}
\end{equation} \noindent Here $\mathbf{B}$ is an antisymmetric rank 2 CT whose
elements which may be arranged into a $3\times3$ matrix. This tensor has 3
independent elements. If these elements transform under rotations like
spherical harmonics, then $\mathbf{B}$ corresponds to a spherical tensor
$S^{(1)}$ of rank 1.  $\mathbf{C}$ is a second-rank, traceless symmetric CT.
Such a CT has 5-independent elements.  If these elements transform as spherical
harmonics, then $\mathbf{C}$ corresponds to a spherical tensor,$S^{(2)}$ of
rank 2.  Finally, $\mathbf{A}$ is a second rank tensor with only one
independent element, and is invariant under transformations (usually defined as
a function of the trace of $\mathbf{T}$ and the diagonal elements of
$\mathbf{C}$). Thus $\mathbf{A}$ will correspond to a ST $S^{(0)}$ of rank 0.\\

\noindent Therefore, even if a CT such as  $\mathbf{T}$ cannot itself be
represented by an ST of a given rank, it can be represented by a linear
combination of STs of varying rank, e.g.,
\begin{equation*}
\mathbf{T} =
  a_{0}S^{(0)}_{0}
 +\sum_{m=-1}^{1}b_{m}S_{m}^{(1)}
 +\sum_{m=-2}^{2}c_{m}S_{m}^{(2)}
\end{equation*}
\begin{equation}
 =\sum_{l=0}^{2}\sum_{m=-l}^{+l}x^{(l)}_{m}S_{m}^{(l)}
\end{equation}
where $a_{0}$, $\{b_{m}\}_{m=-1,0,1}$  and $\{c_{m}\}_{m=-2,-1,0,1,2}\}$ are
the coefficients corresponding to the tensors $\mathbf{A}$, $\mathbf{B}$ and
$\mathbf{C}$.  The coefficients $\{x_{m}^{l}\}$ is just a more compact way of
writing this, and can be compared to equation A6 in Van den Heuvel \& Soncini,
J. Chem. Phys. 138, 054113 (2013).\\

\noindent This is directly connected to the preceeding discussion of
decomposition of spaces formed from the product of angular momenta into linear
combinations of spaces corresponding to a singular angular momenta. The
elements of a Cartesian tensor operator (CTO) represented on spaces
corresponding to a single angular momenta will transform as spherical
harmonics, enabling us to represent these CTOs as a linear combination of
spherical tensor operators (STOs).\\

\noindent To see how the coefficients $a_{0}$, $\{b_{m}\}_{m=-1,0,1}$ and 
 $\{c_{m}\}_{m=-2,-1,0,1,2}\}$ may be
obtained we need to consider how to map between a Cartesian basis $\{x,y,z\}$
and spherical harmonics. First note transform from Cartesian co-ordinate to 
spherical ones; 
\begin{equation}
cos \theta = \frac{z}{r} \text{\ \ \ }
sin \theta = \frac{\sqrt{x^{2}+y^{2}}}{r} \text{\ \ \ }
e^{i\phi} sin\theta  = \frac{x+iy}{r} 
\end{equation}
\noindent Then use these definitions to rewrite the spherical harmonics for
$l=1$ using Cartesian co-ordinates:
\begin{equation}
\Upsilon_{1,-1} = \sqrt{\frac{3}{4\pi}}\frac{x-iy}{r\sqrt{2}} \text{\ \ \ }
\Upsilon_{1,0} = \sqrt{\frac{3}{4\pi}}\frac{z}{r} \text{\ \ \ }
\Upsilon_{1,+1} = \sqrt{\frac{3}{4\pi}}\frac{x+iy}{r\sqrt{2}}
\end{equation}
and $l=2$:
\begin{equation*}
\Upsilon_{2,+2} = \sqrt{\frac{15}{32\pi}}\frac{(x+iy)^{2}}{r^{2}} \text{\ \ \ }
\Upsilon_{2,-2} = \sqrt{\frac{15}{32\pi}}\frac{(x-iy)^{2}}{r^{2}}
\end{equation*}
\begin{equation*}
\Upsilon_{2,+1} = -2\sqrt{\frac{15}{32\pi}}\frac{x+iy}{r^{2}}z \text{\ \ \ }
\Upsilon_{2,-1} = 2\sqrt{\frac{15}{32\pi}}\frac{x-iy}{r^{2}}z
 \end{equation*}
\begin{equation}
\Upsilon_{2,0} = \frac{1}{\sqrt{3}}
 \sqrt{\frac{15}{32\pi}}\Bigg( 3 \frac{z^{2}}{r^{2}} -1 \Bigg )
\end{equation}
\noindent Without loss of generality we can set $r=1$ 
\footnote{defining the Cartesian basis such that $|\mathbf{x}|^{2}+|\mathbf{y}|^{2}+|\mathbf{z}|^{2}$ }. Furthermore, spherical Harmonics of different 
l do not mix under transformations, therefore, the factors common to
all members of a given l can be absorbed into the coefficients, leaving 
us with the much nicer definitions
\begin{equation}
\Upsilon_{1,-1} = \frac{x-iy}{\sqrt{2}} \text{\ \ \ }
\Upsilon_{1,0}  = z \text{\ \ \ }
\Upsilon_{1,+1} = \frac{x+iy}{\sqrt{2}}
\end{equation}
and $l=2$:
\begin{equation*}
\Upsilon_{2,+2} = (x+iy)^{2}\text{\ \ \ }
\Upsilon_{2,-2} = (x-iy)^{2}
\end{equation*}
\begin{equation*}
\Upsilon_{2,+1} = 2(x+iy)z \text { \ \ \ }
\Upsilon_{2,-1} = -2(x-iy)z 
\end{equation*}
\begin{equation}
 \Upsilon_{2,0} =\frac{1}{\sqrt{3}} \Bigg( 3\frac{z^{2}}{r^{2}} -1 \Bigg )
\end{equation}


\section{Appendix}
First, the  spin operators represented in the spin product basis (these are the matrices
used by EasySpin).
\begin{equation}
S(x,1) = 
\begin{bmatrix}
0                  & \frac{1}{\sqrt{2}} & 0\\
\frac{1}{\sqrt{2}} & 0                 & \frac{1}{\sqrt{2}} \\ 
0                  & \frac{1}{\sqrt{2}} & 0
\end{bmatrix}
\end{equation}
\begin{equation}
S(y,1) = 
\begin{bmatrix}
0                  & \frac{i}{\sqrt{2}} & 0\\
\frac{-i}{\sqrt{2}} & 0                 & \frac{-i}{\sqrt{2}} \\ 
0                  & \frac{i}{\sqrt{2}} & 0
\end{bmatrix}
\end{equation}
\begin{equation}
S(z,1) = 
\begin{bmatrix}
-1                  & 0 & 0\\
0 & 0               & 0    \\ 
0                   &   & 1
\end{bmatrix}
\end{equation}
Now for the product operators, $\hat{s}_{i}\otimes\hat{s}_{j}$, represented in the
spin product basis.
\begin{equation}
S_{xx} = 
\begin{bmatrix}
\frac{1}{2} & 0  & \frac{1}{2} \\ 
0           & 1 & 0\\
\frac{1}{2} & 0  & \frac{1}{2} 
\end{bmatrix}
\end{equation}

\begin{equation}
s_{xy} = 
\begin{bmatrix}
\frac{-i}{2} & 0  & \frac{i}{2} \\ 
0           & 1 & 0\\
\frac{-i}{2} & 0  & \frac{i}{2} 
\end{bmatrix}
\end{equation}

\begin{equation}
S_{xz} = 
\begin{bmatrix}
0           & 0 & 0\\
\frac{1}{2} & 0  & \frac{1}{2} \\ 
0           & 0 & 0\\
\end{bmatrix}
\end{equation}


\begin{equation}
s_{yx}
\begin{bmatrix}
\frac{i}{2} & 0  & \frac{i}{2} \\ 
0           & 1 & 0\\
\frac{-i}{2} & 0  & \frac{-i}{2} 
\end{bmatrix}
\end{equation}

\begin{equation}
s_{yy} = 
\begin{bmatrix}
0           & 0 & 0\\
\frac{1}{2} & 0  & \frac{1}{2} \\ 
0           & 0 & 0\\
\end{bmatrix}
\end{equation}
\begin{equation}
s_{yz} = 
\begin{bmatrix}
0           & 0 & 0\\
\frac{1}{2} & 0  & \frac{1}{2} \\ 
0           & 0 & 0\\


\frac{1}{2}  & 0  & \frac{-1}{2} \\ 
0            & 1  & 0\\
\frac{-1}{2} & 0  & \frac{1}{2} 
\end{bmatrix}
\end{equation}

\begin{equation}
s_{zx}
\begin{bmatrix}
\frac{i}{2} & 0  & \frac{i}{2} \\ 
0           & 1 & 0\\
\frac{-i}{2} & 0  & \frac{-i}{2} 
\end{bmatrix}
\end{equation}

\begin{equation}
s_{zy} = 
\begin{bmatrix}
0     &  \frac{-i}{\sqrt{2}} & 0 \\ 
0     & 0                    & 0 \\
0     &  \frac{-i}{\sqrt{2}} & 0 \\
\end{bmatrix}
\end{equation}
\begin{equation}
s_{zz} = 
\begin{bmatrix}
1     & 0 & 0\\
0     & 0 & 0 \\ 
0     & 0 & 1\\
\end{bmatrix}
\end{equation}



\end{document}
