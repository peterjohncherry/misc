\documentclass[12pt]{article}
\usepackage[utf8x]{inputenc}
\usepackage[english]{babel}
\usepackage[T1]{fontenc}
\usepackage{wrapfig}
\usepackage{amsmath}
\usepackage{amssymb}
\usepackage[font=small,format=plain,labelfont=bf,up,textfont=it,up]{caption}
\usepackage{calc}
\usepackage{ifthen}
\usepackage{relsize}
\usepackage{bbold}
\usepackage{mathtext}
\usepackage{pdfpages}
\usepackage{geometry}
 \geometry{
 a4paper,
 total={170mm,257mm},
 left=35mm,
 top=25mm,
 bottom=25mm,
 right=20mm,
 }

\begin{document}
\noindent I think of a vector space, $V$, as being defined by a map.
\begin{equation}
V : E \rightarrow \mathbb{C}^{n}
\end{equation}
\noindent each $\mathbf{v} \in V$, the tuple $c\in \mathbb{C}$ used to represent
$v$ on $E$.\\

\noindent To determine the components of reprentation, i.e., the tuple, $c$, corresponding
to a given vector, $v$, we can define a function, $v(e)$,
which maps from a member, $e \in E$, to a number $c \in \mathbb{C}$.
\begin{equation}
v :  e \rightarrow c \text{ \  \  \ \ \ \ \ \ \ \ \ \ \ \ \ \ }  e\in E \text{ \ \ }c \in \mathbb{C} \text{\ \ \ } v \in V 
\end{equation}
\noindent To see why, consider a vector $\mathbf{v}$ defined on a basis $E$
\begin{equation}
\mathbf{v} = \sum_{i=1}^{dim(E)} v(\mathbf{e}_{i}) \mathbf{e}_{i} = \sum_{i=1}^{dim(E)} v_{i} \mathbf{e}_{i}
\end{equation}
\noindent In other words, $v_{i}$, is just shorthand for $v(\mathbf{e}_{i})$.\\

\noindent As an example take a 3-dimensional vector, $\mathbf{w}$, represented
on the Cartesian basis, $E = \{\mathbf{x},\mathbf{y},\mathbf{z}\}$ :
\begin{equation}
\mathbf{w} = w(\mathbf{x})\mathbf{x}+w(\mathbf{y})\mathbf{y}+w(\mathbf{z})\mathbf{z}
 = w_{x}\mathbf{x}+w_{y}\mathbf{y}+w_{z}\mathbf{z}
\end{equation}
\noindent note the coeffients $\{w_{x},w_{y},w_{z}\}\in \mathbb{C}^{3}$, although
if we are talking about Cartesian vectors we assume the values of these coefficients
to be real, so the tuple is in $\mathbb{R}^{3}$.\\

\noindent Usually we think of $v(e_{i})$, as being a projection of
vector $\mathbf{v}$ represented on , onto the onto the basis functions $\mathbf{e}$. 
Essentially, the dual, $v*$, of $v$, is defined as the tuple of functions which map 
from the members of the basis, $E$, to the coefficients of $v$ represented on this basis.\\

\noindent A vector operator is one which that transforms like a vector.
For example, if we have a vector operator $a\in A$, we can
define it on an basis $B$, were the members of $\hat{b}$ are 
themselves operators, e.g., for a Cartesian vector operator
\begin{equation}
\mathbf{\hat{a}}  = a_{x}\hat{b}_{x}+a_{y}\hat{b}_{y}+a_{z}\hat{b}_{z}
\end{equation}
here $\{a_{i}\}_{i=x,y,z}$ are just coefficients. A rotation, $\hat{R}$, of the
co-ordinate system on which $\hat{a}$ 
\begin{equation}
\hat{a}' = \hat{R}\hat{a}
\end{equation}
can be effected by a transformation of the components $\{a_{i}\}_{i=x,y,z}$
\begin{equation}
\hat{a}_{i'} = \sum_{j=x,y,z} R_{ij} a_{j}
\end{equation}
\noindent where $R_{ij}$ is are the components of the operator used to
effect rotations on the space Cartesian basis $\{x,y,z\}$. \\



then a rotation on the space $$

 which can be defined on a vector
space $E = {e_{x},e_{y},e_{z}}$, then applying a rotation \hat{R} to members of the basis
\begin{equation}

\edn
$$
the members of the 
\begin{equation}
\hat{a} = \sum_{i=x,y,z} \hat{a}_{x}
\end{equation}
\
 In a similar vein, we could define a vector operator, $\tilde{A}$, which
takes two vectors. 

 with
coefficients $\{\tilde{a}_{i}\}_{i=x,y,z}$. Now define a member of $a\in A$ represented
on the space of
 all complex, Hermtian, 2$\times$2 matrics, $H_{2}(\mathbf{C})$. 
\begin{equation}
[a]_{H_{2\times 2}} = 
\begin{bmatrix}

\end{bmatrix}
\end{equation}

 
\noindent So the components of a representation, $[v]_{E}$ of a vector $v\in V$  on a basis $E$ is 
\begin{equation}
[v]_{E} = \sum v_{i}(e_{i}) = c_{i}
\end{equation}

\noindent The map corresponding to an n-dimensional vector takes use from one space, $E$, (which can
be thought of as the basis), to the space of all $n$-tuples of complex numbers, $\mathbb{C}^{n}$,
which can be thought of as the coefficients of the vectors.
So
\begin{equation}
v(e^{i}) \rightarrow v_{i}
\end{equation}
For reasons of simplicity 

\begin{equation}
v(e_{i}) = 
\end{equation}


\end{document}
