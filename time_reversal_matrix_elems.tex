\documentclass[12pt]{article}
\usepackage[utf8x]{inputenc}
\usepackage[english]{babel}
\usepackage[T1]{fontenc}
\usepackage{wrapfig}
\usepackage{amsmath}
\usepackage{amssymb}
\usepackage[font=small,format=plain,labelfont=bf,up,textfont=it,up]{caption}
\usepackage{calc}
\usepackage{ifthen}
\usepackage{relsize}
\usepackage{bbold}
\usepackage{mathtext}
\usepackage{pdfpages}
\usepackage{geometry}
 \geometry{
 a4paper,
 total={170mm,257mm},
 left=35mm,
 top=25mm,
 bottom=25mm,
 right=20mm,
 }
\newcommand{\K}{\mathlarger{\mathlarger{\hat{\kappa}}}} 

\begin{document}
\noindent Define zeroth order Hamiltonian, wavefunction etc.,
\begin{equation}
H^{(0)}\Psi_{k}^{(0)} = E_{k}\Psi_{k}^{(0)}
\end{equation}
\noindent Define the one electron time reversal operator, $\hat{\kappa}$:
\begin{equation}
\hat{\kappa} = -i \hat{C}\sigma_{y}
\end{equation}
\noindent where $\hat{C}$ is the complext conjugation operator, and
$\sigma_{y}$ is the $y$ Pauli matrix. The $N$-electron time reversal operator
is just defined as 
\begin{equation*}
\K = 
(\hat{\kappa}\otimes 1 \otimes ... \otimes 1)
\end{equation*}
\begin{equation*}
\oplus
(1 \otimes \hat{\kappa}\otimes  ... \otimes 1)
\end{equation*}
\begin{equation}
\oplus
...
(\otimes 1 \otimes  ... \otimes \hat{\kappa})
\end{equation}
\noindent Have Kramer's pair relations
\begin{equation}
\K \Psi_{k} = \overline{\Psi}_{k}
\K \overline{\Psi}_{k} = \K\K\Psi_{k} = (-1)^{N_{e}} \Psi_{k}
\end{equation}
\noindent i.e., $\Psi$ and $\overline{\Psi}$ form a doublet.  The dependence
of the sign on the number of electrons, $N_{e}$, is important. The above relation 
holds in both a relativistic and non-relativistic framework.\\

\noindent Note that the zeroth order Hamiltonian, $\hat{H}^{(0)}$ commutes with the time reversal operator, $\K$.
\begin{equation}
[\K, \hat{H} ] = 0 
\end{equation}
\noindent However, the Zeeman Hamiltonian, $\hat{H}^{Zee}$, anti-commutes:
\begin{equation}
\{\K, \hat{H} \} = 0 
\end{equation}
\noindent 

\noindent We can use this property of the time reversal operator to demonstrate
that $\Psi$ is orthorgonal to its Kramers pair, $\overline{\Psi}$\\
\begin{equation}
\langle \Psi | \overline\Psi  \rangle=
\langle \Psi |\K \Psi  \rangle=
\langle \Psi | \K^{\dagger}\K\K\Psi  \rangle=
\langle \overline{\Psi} | \K^{2}\Psi  \rangle=
-\langle \overline{\Psi} | \Psi  \rangle
\end{equation}
\begin{equation}
\langle \Psi | \overline\Psi  \rangle=
-\langle \overline{\Psi} | \Psi  \rangle 
\text{ \ \ \ \ } \rightarrow \text{ \ \ \ \ }
\langle \Psi | \overline\Psi  \rangle= 0
\end{equation}
\noindent If $\Psi$ and $\overline{\Psi}$ are orthogonal they
must be different states.\\

\noindent We can use these relations to determine some important 
properties of the spin Hamiltonian
used to describe a doublet system with an odd number of electrons.
Consider the case of  the Zeeman Hamiltonian $\hat{H^{Zee}}$;
 a one electron operator which is anti-symmetric under time reversal 
\begin{equation*}
g_{uv}\sigma_{u} =
[\hat{H}_{u}^{Zee}]_{\{\Psi, \overline{\Psi}\}} = 
\begin{bmatrix}
\langle \Psi | \hat{H}^{Zee}_{u} |\Psi  \rangle &
\langle \Psi | \hat{H}^{Zee}_{u} | \overline{\Psi}  \rangle
\\ 
\langle \overline{\Psi} | \hat{H}^{Zee}_{u} | \Psi  \rangle &
\langle \overline{\Psi} | \hat{H}^{Zee}_{u} | \overline\Psi  \rangle
\end{bmatrix}
\end{equation*}
\begin{equation*}
=
\begin{bmatrix}
\langle \Psi | \hat{H}^{Zee}_{u}   | \Psi  \rangle &
\langle \Psi | \hat{H}^{Zee}_{u}\K | \Psi  \rangle
\\ 
\langle \Psi |\K^{\dagger} \hat{H}^{Zee}_{u} | \Psi  \rangle &
\langle \Psi |\K^{\dagger} \hat{H}^{Zee}_{u} \K | \Psi  \rangle
\end{bmatrix}
\end{equation*}
\noindent First consider the diagonal elements. Noting
that $\hat{H}^{Zee}_{u}$ is antisymmetric under time 
reversal, i.e., $\K^{\dagger} \hat{H}^{Zee}_{u} \K = - \hat{H}^{Zee}_{u}$,
we find 
\begin{equation}
\langle \Psi | \hat{H}^{Zee}_{u}   | \Psi  \rangle 
= -\langle \overline \Psi | \hat{H}^{Zee}_{u} | \overline \Psi  \rangle.
\end{equation}
\noindent In the case of the off diagonal elements we can
use the fact that, for odd numbered electrons, $\K^{dagger}\K = 1$,
and that $\K$ anticommutes with the Hamiltonian . Which
when inserted into one of the BraKets yields 
\begin{equation}
\langle \Psi | \hat{H}^{Zee}_{u} | \overline{\Psi}  \rangle
=
\langle \Psi | \hat{H}^{Zee}_{u} \K| \Psi  \rangle
=
\langle \Psi | \K^{\dagger} \hat{H}^{Zee}_{u} | \Psi  \rangle^{*} 
=
\langle \overline{\Psi} | \hat{H}^{Zee}_{u} | \Psi \rangle^{*} 
\end{equation}
\noindent So the spin Hamiltonian is Hermitian. This means it is possible to
represent the spin Hamiltonian using only the three Pauli matrices, $\sigma_{y}$,
$\sigma_{x}$, and $\sigma_{z}$.\\

\noindent Things become more complicated when we consider 
doublet systems where there is an even number of electrons. In this case
\begin{equation}
\K^{2}|\Phi\rangle = |\Phi\rangle
\end{equation}
\noindent Begin by considering the case of a triplet (two unpaired electrons).
We shall begin by taking the case of a triplet state, where the tree states
are defined as
\begin{equation}
\Psi_{1,1} = | \alpha, \alpha\rangle
\end{equation}
\begin{equation}
\Psi_{1,0} = \frac{1}{\sqrt{2}}|( \alpha\beta\rangle -| \beta\alpha\rangle)
\end{equation}
\begin{equation}
\Psi_{1,-1} = | \beta, \beta\rangle
\end{equation}
\noindent where the left subscript labels the multiplet, and the
right subscript the projection of spin angualr momentum on the z-axis.
The representation of the one-electron Zeeman operator in this basis is
\begin{equation}
\begin{bmatrix}
\langle 1,1 | \hat{H}^{Zee}_{u} | 1,1  \rangle&
\langle 1,1 | \hat{H}^{Zee}_{u} | 1,0  \rangle&
\langle 1,1 | \hat{H}^{Zee}_{u} | 1,-1  \rangle
\\ 
\langle 1,0 | \hat{H}^{Zee}_{u} | 1,1    \rangle&
\langle 1,0 | \hat{H}^{Zee}_{u} | 1,0    \rangle&
\langle 1,0 | \hat{H}^{Zee}_{u} | 1,-1   \rangle
\\ 
\langle 1,-1 | \hat{H}^{Zee}_{u} | 1,1    \rangle& 
\langle 1,-1 | \hat{H}^{Zee}_{u} | 1,0    \rangle&
\langle 1,-1 | \hat{H}^{Zee}_{u} | 1,-1   \rangle
\end{bmatrix}
\end{equation}
\begin{equation}
=
\begin{bmatrix}
\langle 1,1 |            \hat{H}^{Zee}_{u}   | 1,1  \rangle&
\langle 1,1 |            \hat{H}^{Zee}_{u}   | 1,0  \rangle&
\langle 1,1 |            \hat{H}^{Zee}_{u}\K | 1,1  \rangle
\\                       
\langle 1,0 |            \hat{H}^{Zee}_{u}   | 1,1   \rangle&
\langle 1,0 |            \hat{H}^{Zee}_{u}   | 1,0   \rangle&
\langle 1,0 |            \hat{H}^{Zee}_{u}\K | 1,1   \rangle
\\ 
\langle 1,1 |\K^{\dagger} \hat{H}^{Zee}_{u}   | 1,1   \rangle& 
\langle 1,1 |\K^{\dagger} \hat{H}^{Zee}_{u}   | 1,0   \rangle&
\langle 1,1 |\K^{\dagger} \hat{H}^{Zee}_{u}\K | 1,1   \rangle
\end{bmatrix}
\end{equation}
First we have
\begin{equation}
\langle 1,1 |            \hat{H}^{Zee}_{u}   | 1,1  \rangle
=\langle 1,1 |\K^{\dagger} \hat{H}^{Zee}_{u}\K | 1,1   \rangle
=-\langle 1,1 |\hat{H}^{Zee}_{u}| 1,1  \rangle
\end{equation}
\noindent We can also note 
\begin{equation*}
\langle 1,1 | \hat{H}^{Zee}_{u} | 1,-1  \rangle
=
  \langle 1,1 |                  \hat{H}^{Zee}_{u}\K | 1,1  \rangle
\end{equation*}
\begin{equation}
= \langle 1,1 |\K^{\dagger}\K^{\dagger} \hat{H}^{Zee}_{u}\K   | 1,1  \rangle
= - \langle 1,-1 |\hat{H}^{Zee}_{u} | 1,1  \rangle
\end{equation}
\noindent In order to simplify things we can use the transformation
\begin{equation}
\frac{1}{\sqrt{2}}
\begin{bmatrix}
1 &  i \\ 
1 & -i 
\end{bmatrix}
\begin{bmatrix}
\Psi_{1}\\
\Psi_{-1}
\end{bmatrix}
\end{equation}
\end{document}
